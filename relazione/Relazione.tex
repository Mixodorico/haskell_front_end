\documentclass[12pt]{article}
\usepackage[utf8x]{inputenc}

\begin{document}

\begin{center}
\noindent{\Huge Progetto LC2, Parte 3}
\end{center}

\begin{center}
\noindent{\Huge Gruppo 1}
\end{center}

\section*{Assunzioni}
\begin{itemize}
\item Non è possibile dichiarare più di una funzione con lo stesso identificativo.
\item non è consientito dichiarare delle variabili con lo stesso identificativo nello stesso blocco, mentre è possibile se avviene su blocchi (e sottoblocchi) diversi;
\item è stato usato l'operatore '!' nelle condizioni degli 'if per veicolare correttamente i salti condizionati;
\item nel corpo delle procedure non deve presentare l'istruzione return, sempre richiesta invece nei corpi delle funzioni;
\item sono stati mantenuti solamente gli usuali comandi per il controllo della sequenza (\textit{condizionali semplici}, \textit{iterazione indeterminata});
\item le guardie booleane dei controlli di sequenza vengono gestite tramite short-cut mentre le altre espressioni booleani vengono gestite senza short-cut.
\end{itemize}

\subsection*{Scelte implementative}
\begin{itemize}
\item Abbiamo generato \textit{lexer} e \textit{parser} tramite il tool BNFC, partendo da una gramatica iniziale;
\item la gestione del \textit{Type Checker} e del \textit{Three Address Code} vengono fatte all'interno del parser;
\item le funzioni \textit{write} sono trattate come statements che prendendono come argomento una right expression, mentre le funzioni \textit{read} vengono viste come right expressions e hanno una lista di parametri vuota.
\end{itemize}

\subsection*{Type Checker}
Il Type Checker viene gestito all'interno del parser appoggiandosi ad un modulo esterno (\textit{nome-modulo}) contenente una serie di funzioni di controllo sull'environment. Il Type Checker tiene conto in modo automatico di una conversione dal tipo int al tipo float.
In particolare, 

\subsection*{Three Address Code}
Il Three Address Code viene generato interamente nel parser, utilizzando un modulo esterno (\textit{nome-modulo}) contenente la struttura dati per gestirne la generazione ed una funzione addetta alla stampa.
Il Three Address Code viene rappresentato dalla 

\end{document}
