\documentclass[12pt]{article}
\usepackage[utf8x]{inputenc}

\begin{document}

\begin{center}
\noindent{\Huge Progetto LC2, Parte 3}
\end{center}

\begin{center}
\noindent{\Huge Gruppo 1}
\end{center}

\section*{Assunzioni}
\begin{itemize}
\item Non è possibile dichiarare più di una funzione con lo stesso identificatore.
\item non è consientito dichiarare delle variabili con lo stesso identificatore nello stesso blocco, mentre è possibile se avviene su blocchi (e sottoblocchi) diversi;
\item è stato usato l'operatore `!' nelle condizioni degli if per veicolare correttamente i salti condizionati;
\item nel corpo delle procedure non deve presentare l'istruzione return, sempre richiesta invece nei corpi delle funzioni;
\item sono stati implementati solamente i comandi per il controllo della sequenza richiesti nel testo del progetto (\textit{condizionali semplici}, \textit{iterazione indeterminata});
\item le guardie booleane dei controlli di sequenza vengono gestite tramite short-cut mentre le altre espressioni booleani vengono gestite senza short-cut.
\end{itemize}

\subsection*{Scelte implementative}
\begin{itemize}
\item Abbiamo generato \textit{lexer} e \textit{parser} tramite il tool BNFC, partendo da una gramatica iniziale;
\item la gestione del \textit{Type Checker} e del \textit{Three Address Code} vengono fatte all'interno del parser;
\item le funzioni \textit{write} sono trattate come statements che prendendono come argomento una right expression, mentre le funzioni \textit{read} vengono viste come right expressions e hanno una lista di parametri di input vuota.
\end{itemize}

\subsection*{Type Checker}
Il Type Checker viene gestito all'interno del parser. Si tiene conto in modo automatico di una conversione dal tipo int al tipo float.

\subsection*{Three Address Code}
Il Three Address Code viene generato interamente nel parser, utilizzando un modulo esterno (\textit{TAC.hs}) contenente la struttura dati per gestirne la generazione e le funzioni addette alla stampa.

\subsection*{Test Case}
Sono stati preparati dei test-case significativi che possono essere eseguiti in sequenza attraverso il comando \textit{make demo}, oppure singolarmente attraverso i comandi \textit{make demo1}, \textit{make demo2}, \textit{make demo3}.

\end{document}
